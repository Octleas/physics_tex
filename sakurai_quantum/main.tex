% !TEX program = lualatex
\documentclass[11pt, a4paper]{ltjsarticle}
    \usepackage{amsmath}
    \usepackage{amssymb}
    \usepackage{amsthm}
    \usepackage{graphicx}
    \usepackage{mathcomp}
    \usepackage{float}
    \usepackage{bm}
    \usepackage{ascmac}
    \usepackage{booktabs}
    \usepackage{physics}
    \usepackage{tikz}
    \usepackage{multirow}
    \usepackage{mathtools}
    \usepackage{caption}
    \usepackage{siunitx}
    \usepackage{lmodern}
    \usepackage{multicol}

    % レイアウト調整
    \setlength{\textwidth}{\fullwidth}
    \setlength{\textheight}{40\baselineskip}
    \addtolength{\textheight}{\topskip}
    \setlength{\voffset}{-0.6in}
    \setlength{\headsep}{0.3in}
    \setlength{\parindent}{1em}
    \setlength{\parskip}{0.3em}

    % コマンド定義
    \newcommand{\zu}[3][10cm]{%
      \begin{figure}[H]
            \centering
            \includegraphics[width=#1]{#2}
            \caption{#3}
        \end{figure}%
    }

    \theoremstyle{plain}
    \newtheorem{thm}{Theorem}[section]

\title{Modern Quantum Mechanics}
\author{J. J. SAKURAI}
\date{\today}

\begin{document}
\maketitle
\section{[1.4.5 Exercise] スピン$1/2$系における不確定性関係の検証}
不確定性関係の一般式は
\begin{equation}
    \langle(\Delta A)^2\rangle \langle(\Delta B)^2\rangle \geq \frac{1}{4}|\langle[A, B]\rangle|^2
\end{equation}
$S_z$の固有状態$|+z\rangle$における各演算子の期待値を求める.
\begin{eqnarray}
    \langle S_x \rangle &=& \langle +z|S_x|+z \rangle = 0\\
    \langle S_y \rangle &=& \langle +z|S_y|+z \rangle = 0
\end{eqnarray}
それぞれの分散は$\langle S_i^2 \rangle = \frac{\hbar^2}{4}$より自明にわかるから, 
\begin{equation}
    ((1)LHS) = \langle (\Delta S_x)^2 \rangle\langle (\Delta S_y)^2 \rangle = \frac{\hbar^2}{4}\cdot \frac{\hbar^2}{4} = \left(\frac{\hbar^2}{4}\right)^2
\end{equation}
また,
\begin{equation}
    \langle [S_x,S_y] \rangle = \langle i\hbar S_z \rangle = i\hbar \langle +z|S_z|+z \rangle = i\hbar (\frac{\hbar}{2}) = \frac{i\hbar}{2}
\end{equation}
より, RHSは
\begin{equation}
    \frac{1}{4}|\
\end{equation}

\end{document}
