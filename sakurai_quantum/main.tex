% !TEX program = lualatex
\documentclass[11pt, a4paper]{ltjsarticle}
    \usepackage{amsmath}
    \usepackage{amssymb}
    \usepackage{amsthm}
    \usepackage{graphicx}
    \usepackage{mathcomp}
    \usepackage{float}
    \usepackage{bm}
    \usepackage{ascmac}
    \usepackage{booktabs}
    \usepackage{physics2}
    \usephysicsmodule{ab}
    \usephysicsmodule{braket}
    \usepackage{tikz}
    \usepackage{multirow}
    \usepackage{mathtools}
    \usepackage{caption}
    \usepackage{siunitx}
    \usepackage{lmodern}
    \usepackage{multicol}

    % レイアウト調整
    \setlength{\textwidth}{\fullwidth}
    \setlength{\textheight}{40\baselineskip}
    \addtolength{\textheight}{\topskip}
    \setlength{\voffset}{-0.6in}
    \setlength{\headsep}{0.3in}
    \setlength{\parindent}{1em}
    \setlength{\parskip}{0.3em}

    % コマンド定義
    \newcommand{\zu}[3][10cm]{%
      \begin{figure}[H]
            \centering
            \includegraphics[width=#1]{#2}
            \caption{#3}
        \end{figure}%
    }

    \renewcommand{\ket}[1]{\lvert#1\rangle}
    \renewcommand{\bra}[1]{\langle#1\rvert}
    \renewcommand{\braket}[2]{\langle{#1}\vert{#2}\rangle}

    \theoremstyle{plain}
    \newtheorem{thm}{Theorem}[section]

\title{Modern Quantum Mechanics}
\author{J. J. SAKURAI}
\date{\today}

\begin{document}
\maketitle
\section{[1.4.5 Exercise] スピン$1/2$系における不確定性関係の検証}
不確定性関係の一般式は
\begin{equation}
    \langle(\Delta A)^2\rangle \langle(\Delta B)^2\rangle \geq \frac{1}{4}|\langle[A, B]\rangle|^2
\end{equation}
$S_z$の固有状態$\ket{+z}$における各演算子の期待値を求める.
\begin{eqnarray}
    \langle S_x \rangle &=& \langle +z|S_x|+z \rangle = 0\\
    \langle S_y \rangle &=& \langle +z|S_y|+z \rangle = 0
\end{eqnarray}
それぞれの分散は$\langle S_i^2 \rangle = \frac{\hbar^2}{4}$より自明にわかるから, 
\begin{equation}
    ((1)LHS) = \langle (\Delta S_x)^2 \rangle\langle (\Delta S_y)^2 \rangle = \frac{\hbar^2}{4}\cdot \frac{\hbar^2}{4} = \ab(\frac{\hbar^2}{4})^2
\end{equation}
また,
\begin{equation}
    \langle [S_x,S_y] \rangle = \langle i\hbar S_z \rangle = i\hbar \langle +z|S_z|+z \rangle = i\hbar (\frac{\hbar}{2}) = \frac{i\hbar}{2}
\end{equation}
より, 
\begin{equation}
    ((1)RHS) = \frac{1}{4}|\langle [S_x,S_y] \rangle|^2 = \frac{1}{4}\ab|\frac{i\hbar^2}{2}|^2 = \ab(\frac{\hbar^2}{4})^2
\end{equation}
よって$A = S_x$, $B = S_y$のときに不確定性関係が成り立つことが示された.
\section{ユニタリ演算子(unitary operator)}
\begin{itembox}[l]{定理 3.}
    正規直交性と完全性を満たす2組の基底ケットが与えられている. このときユニタリ演算子$U$が存在して
    \begin{equation}
        \ket{b^{(1)}} = U\ket{a^{(1)}}, \ket{b^{(2)}} = U\ket{a^{(2)}}, \dots, \ket{b^{(N)}} = U\ket{a^{(N)}}
    \end{equation}
    が成立する. ここで, ユニタリ演算子とは, 条件
    \begin{equation}
        U^\dagger U = 1
    \end{equation}
    および
    \begin{equation}
        UU^\dagger = 1
    \end{equation}
    を満たす演算子.
\end{itembox}
証明:\\
\begin{equation}
    U = \sum_k \ket{b^{(k)}}\bra{a^{(k)}}
\end{equation}
という演算子が$U$の役割を果たすものとし, この$U$を$\ket{a^{(l)}}$に掛け算すると,
\begin{equation}
    U\ket{a^{(l)}} = \sum_k \bra{b^{(k)}}\braket{a^{(k)}}{a^{(l)}}
\end{equation}
$k=l$のときのみ, $\braket{a^{(k)}}{a^{(l)}} = 1$だから, 
\begin{equation}
    U\ket{a^{(l)}} = \bra{b^{(l)}}
\end{equation}
となることが$\{\ket{a^\prime}\}$の正規直交性から保証される.
ちなむと
\begin{equation}
    U^\dagger U = \sum_k \sum_l \ket{a^{(l)}}\braket{b^{(l)}}{b^{(k)}}\bra{a^{(k)}} = \sum_k \ket{a^{(k)}}\bra{a^{(k)}} = 1
\end{equation}
\end{document}

