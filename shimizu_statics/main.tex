% !TEX program = lualatex
\documentclass[11pt, a4paper]{ltjsarticle}
    \usepackage{amsmath}
    \usepackage{amssymb}
    \usepackage{amsthm}
    \usepackage{graphicx}
    \usepackage{mathcomp}
    \usepackage{float}
    \usepackage{bm}
    \usepackage{ascmac}
    \usepackage{booktabs}
    \usepackage{physics}
    \usepackage{tikz}
    \usepackage{multirow}
    \usepackage{mathtools}
    \usepackage{caption}
    \usepackage{siunitx}
    \usepackage{lmodern}
    \usepackage{multicol}

    % レイアウト調整
    \setlength{\textwidth}{\fullwidth}
    \setlength{\textheight}{40\baselineskip}
    \addtolength{\textheight}{\topskip}
    \setlength{\voffset}{-0.6in}
    \setlength{\headsep}{0.3in}
    \setlength{\parindent}{1em}
    \setlength{\parskip}{0.3em}

    % コマンド定義
    \newcommand{\zu}[3][10cm]{%
      \begin{figure}[H]
            \centering
            \includegraphics[width=#1]{#2}
            \caption{#3}
        \end{figure}%
    }

    \theoremstyle{plain}
    \newtheorem{thm}{Theorem}[section]

\title{Principles of Statistical Mechanics}
\author{Akira Shimizu}
\date{\today}

\begin{document}
\maketitle
\section{熱力学}
\begin{itembox}[l]{熱力学の要請 I-(i):平衡状態への移行}
    系を孤立させて(静的な外場だけはあってもよい)十分長いが有限の時間放置すれば、マクロに見て時間変化しない特別な状態へと移行する。このときの系の状態を\textbf{平衡状態}、または\textbf{熱平衡状態}と呼ぶ。
\end{itembox}

\begin{itembox}[l]{熱力学の要請 I-(ii):部分系の平衡状態}
    もしもある部分系の状態が、その部分系をそのまま孤立させた(ただし静的な外場は同じだけかける)ときの平衡状態とマクロに見て同じ状態にあれば、その部分系の状態も\textbf{平衡状態}と呼ぶ。平衡状態にある系の部分系はどれも平衡状態にある。
\end{itembox}

\begin{itembox}[l]{熱力学の要請 II-(i):エントロピーの存在}
    それぞれの平衡状態ごとに値が一意的に定まる、\textbf{エントロピー}という物理量 $S$ が存在する。
\end{itembox}

\begin{itembox}[l]{熱力学の要請 II-(ii):単純系のエントロピー}
    単純系のエントロピー $S$ は、その物質の、エネルギー $E$、体積 $V$、物質量 $N$ の関数である:
    \[ S = S(E, V, N) \]
    これを\textbf{基本関係式}と呼ぶ。単純系の部分系は、元の単純系と同じ基本関係式を持つ。
\end{itembox}

\begin{itembox}[l]{熱力学の要請 II-(iii):基本関係式の解析的性質}
    基本関係式は $E, V, N$ いずれについても偏微分可能であり、しかも、どの偏微分係数も連続である。また、$E$ についての偏微分係数は、正で下限は $0$ で上限はない:
    \[ \pdv{S}{E} > 0 \]
\end{itembox}

\begin{itembox}[l]{熱力学の要請 II-(iv):均一な平衡状態}
    平衡状態にある単純系はそれぞれがマクロに見て空間的に均一な状態にある部分系たちに分割できる(部分系の間の境界はマクロに見て無視できる)。それぞれの均一な部分系の状態は、その部分系の $E, V, N$ の値で一意的にマクロに定まる。また、$E, V, N$ がそれと同じ値を持つような不均一な平衡状態は、その部分系には存在しない。
\end{itembox}

\begin{itembox}[l]{熱力学の要請 II-(v):エントロピー最大の原理}
    複合系は、それを構成するすべての単純系が平衡状態にあって、かつ、与えられた条件の下で、$S$ が最大になる時に、そしてその場合に限り、平衡状態にある。また、平衡状態における複合系のエントロピーは、$S$ の最大値に等しい。
\end{itembox}


\section{平衡統計力学の基本原理A}
\subsection{基本原理を提示するための準備}
そもそもの統計力学の目的として, 基本関係式$S(E,V,N)$をミクロ物理学から求めたい.
その手段については次の章.\\

この本の前提 \rightarrow ミクロな構成要素の間の相互作用が, 平衡状態においては"実効的"に短距離相互作用になっているようなマクロ系.\\
なぜか? \rightarrow 相加性が成り立つ必要があるから.\\

電荷を持つ粒子の間に働くC相互作用は長距離相互作用であるが, プラスとマイナスが混ざり合うことで電荷が打ち消し合い, 短距離相互作用と実効的にみなせる.\\

ただし, 重力は引力しかないため, 天体や銀河のような系は対象外. \\

孤立系で考えるべきか, 開放系で考えるべきか.

\begin{itemize}
    \item 孤立系\rightarrow 外部から一切の相互作用を受けないような系
    \item 開放系\rightarrow 孤立系でない系
\end{itemize}

この本の方針: 統計力学の出発点としては, 孤立系を考え, 着目系全体が静止しているような座標系を選んで議論する.

\subsection{平衡状態を表すミクロ状態}
統計力学における平衡状態とは, 熱力学の平衡状態そのものであるとする.

\begin{itembox}[l]{定理5.1(平衡状態)}
    熱力学の要請Ⅰで定義され, 要請Ⅱを満たすようなマクロ状態を平衡状態(equilibrium state)と呼ぶ.
\end{itembox}

熱力学における平衡状態は$E,V,N$で張られる3次元の熱力学的状態空間の1点に対応する. この状態空間の次元は$N$に依らない. 同じ状態を古典力学で記述しようとすると, $6N$次元の相空間を考える必要がある.

\subsection{古典力学の純粋状態と混合状態}
一般に物理状態は「純粋状態」と「混合状態」という2種類に大別できる.\\
古典力学において, 純粋状態(pure state)は, 相空間の1点で表される状態.\\
混合状態(mixed state)は相空間上に広がった確率分布関数で表される状態. \\
量子力学で学んだ認識でおk

\subsection{よくみかける誤解}
誤解: マクロな一つの平衡状態というのはミクロな一つの混合状態に対応する.\\
もしも混合状態が平衡状態であると定義すると, 観測した瞬間に瞬間に状態が変化する \rightarrow 平衡状態でなくなる.\\
わかりづらいと思うので本文の文章を引用しておく.
\begin{quotation}
    たとえば,あるマクロ系の一つの平衡状態を考え,そのミクロ状態が,この平衡状態で定まる混合状態だとしてみよう.簡単のため,このマクロ系は古典力学に従う粒子より成る気体だとする.すると,全ての粒子の位置と運動量を同時に測定することが,少なくとも原理的には可能である.そういう測定をしたらどうなるかを考えよう.要するに,平衡状態にある系をよく見る,あるいは高解像度の写真を撮ったらどうなるかを考えるわけだ.すると, 測定前には図5.1 (b) のような混合状態だったのが, 測定によって全ての粒子の位置と運動量が(測定誤差の範囲内で)知れるから,測定後のミクロ状態は, 次のいずれかになる:
    \begin{itemize}
    \item 非常に精度の良い測定をした場合\\
    図5.1 (a) のような純粋状態,またはそれに近い混合状態
    \item もっと粗い精度の測定をした場合\\
    上のケースよりは確率分布が広がっているが, 測定前の状態である図5.1 (b) よりは狭まった分布を持つ混合状態
    \end{itemize}
    いずれの場合でも,測定前の混合状態とは異なる状態になっている.しかも,実験をする度に異なる状態になる(確率的にばらつく).したがって,もしも (5.4) が正しければ,もとの平衡状態とは違うマクロ状態になっているはずだ.しかし,膨大な実験や経験によると,平衡状態を見たり写真を撮ったりしただけで別のマクロ状態になるようなことはない.この事実は, そのようなことが起こる確率は圧倒的に小さく, ほとんど100\%の確率で,測定後も同じ平衡状態に留まることを示している.
\end{quotation}
では, 混合状態ではなく純粋状態, すなわち相空間上の一点に対応すればよいのか?\\
もう一つの誤解: マクロな一つの平衡状態というのはミクロな一つの純粋状態に対応する.\\
純粋状態は運動方程式によって時々刻々と変化する. よってこの誤解が正しいのであれば, 平衡状態にあった系はミクロな時間スケールで, 平衡状態ではなくなってしまう. これは実験,経験に反する.

\subsection{平衡状態の典型性}
正解: 一つの平衡状態には, 膨大な数のミクロ状態(純粋状態も混合状態も含む)の集合が対応する. 系において、圧倒的多数(overwhelming majority)な状態が平衡状態, 少数な状態は熱力学極限($V\rightarrow \infty$)を取れば無いも同然.
\begin{equation}
    \frac{E,V,N で指定される平衡状態であるようなミクロ状態の数}{E,V,N がその平衡状態とマクロには同じ値のミクロ状態の総数} \rightarrow 1 
\end{equation}

\subsection{一般の場合}
エントロピーの自然な変数が$E,V,N$でない, すなわちエントロピーの自然な変数が$E$と, いくつかの相加変数$\mathbf{X} \equiv X_1,\dots, X_2$であるような一般の単純系を考える.
そのとき, 平衡状態の典型性は次のような仮説になる:

\begin{itembox}[l]{単純系の平衡状態の典型性}
    エントロピーの自然な変数 $E,\mathbf{X}$ の値を任意に一つ選び, それとマクロに同じ値の $E, \mathbf{X}$ を持つミクロ状態等を全て集めた集合を考えると, そこに含まれるミクロ状態のほとんど全てが, マクロ状態としては, $E, \mathbf{X}$で指定される平衡状態である.
\end{itembox}

\begin{itembox}[l]{平衡統計力学の基本原理A}
    マクロ系(単純系でも複合系でもよい)のミクロ状態のうちの, 与えられた条件(平衡状態を指定するのに十分な相加変数の組の値と束縛条件)をマクロな精度で満たすようなミクロ状態たちを全て集めた集合を考えると, そこに含まれるミクロ状態のほとんど全てが, マクロ状態としては, この条件の下でも平衡状態である. このことを平衡状態の典型性(typicality of equilibrium state)と呼ぶ.
\end{itembox}

平衡状態の典型性を基本原理に採用したことにより, 定理5.1は以下のようになる:
\begin{itembox}[l]{定理5.1 平衡状態を表すミクロ状態}
    与えられた条件(平衡状態を指定するのに十分な相加変数の組の値と束縛条件)の下での平衡状態は, その条件をマクロな制度で満たす膨大な数のミクロ状態から, 特段の(あえて非平衡状態を抜き出すような)バイアスをかけずに選びさえすれば, どのミクロ状態を選んでも(熱力学極限で100\%になる確率で)正しく表せる. このとき, 一つのミクロ状態(純粋状態)だけを選び出してもいいし, 複数このミクロ状態を選んでそれらの混合状態で表しても(どれも熱力学極限で100\%の確率で同じ平衡状態なのだから)正しく表せる.
\end{itembox}

\section{そもそも, Gibbsのアンサンブル形式とは?}
巨視的に同じ条件下にある力学系を無数に集めた仮想的な集団(アンサンブル)を用いて平衡状態を記述する手法.\\
エルゴード仮説 \rightarrow 「1つの系が長時間かけて通過する状態の全て(時間平均)」と、「アンサンブルが瞬間に実現している状態の分布(アンサンブル平均)」は同じ.\\
Gibbsは系が置かれている環境に応じて, 確率分布の形を3つに分類した.
\begin{itemize}
    \item ミクロカノニカル\\
        孤立系$(E,V,N一定)$, エネルギーが$E$の状態は全て同じ確率で現れる(等重率の原理).
    \item カノニカル\\
        温度一定$(T,V,N一定)$, エネルギーが低いほど確率が高い(Boltzmann分布
        $P_i \propto \exp(-\frac{E_i}{k_BT})$).
    \item グランドカノニカル\\
        温度・化学ポテンシャル一定$(T,V,\mu 一定)$, 粒子数によっても確率が変動する.(Gibbs分布$P_i \propto \exp(-\frac{E_i - \mu N_i}{k_B T})$)
\end{itemize}
清水っちは平衡状態の定義にアンサンブルという概念を使うのがキモいと思い(主にコピー, 確率分布の部分が),典型性を用いて平衡状態を定義した.
\end{document}